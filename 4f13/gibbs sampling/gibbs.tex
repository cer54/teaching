\documentclass[10pt]{beamer}
\usefonttheme{professionalfonts,serif}
\def\newblock{\hskip .11em plus .33em minus .07em}
\usepackage[numbers,sort]{natbib}

\usepackage{fontspec} % This line only for XeLaTeX and LuaLaTeX
\setmainfont{Sabon LT Std} % text font Sabon
\usepackage[OT1,euler-digits]{eulervm} % math font euler

%\renewcommand{\rmdefault}{psbx}
%\usepackage[utf8]{inputenc}
%\usepackage[T1]{fontenc}
\usepackage{textcomp}
%\usepackage{eulervm}
\usepackage{bm}

\usetheme{default}           % tips from David Blei
\useinnertheme{circles}
\useoutertheme{infolines}
\setbeamertemplate{headline}{}
\setbeamertemplate{navigation symbols}{}
\setbeamerfont{itemize/enumerate subbody}{size=\normalsize}
\setbeamerfont{itemize/enumerate subsubbody}{size=\normalsize}
\usecolortheme{seahorse}
\setbeamersize{text margin left=2mm,text margin right=2mm}

\graphicspath{{../../figures/}}

\definecolor{mypine}{rgb}{0.05,0.45,0.05}
\definecolor{mycyan}{rgb}{0.0,0.9,0.9}
\newcommand{\Red}{\textcolor{red}}
\newcommand{\Blue}{\textcolor{blue}}
\newcommand{\Green}{\textcolor{mypine}}
\newcommand{\PineGreen}{\textcolor{mypine}}
\newcommand{\Magenta}{\textcolor{magenta}}
\newcommand{\Cyan}{\textcolor{mycyan}}

\newcommand{\N}{\mathcal{N}}
\newcommand{\R}{\mathbb{R}}
\newcommand{\T}{{\scriptsize^{\top}}}
\newcommand{\D}{\mathcal{D}}
\newcommand{\F}{\mathcal{F}}
\newcommand{\E}{\mathbb{E}}
\newcommand{\V}{\mathbb{V}}
\newcommand{\M}{\mathcal{M}}
\newcommand{\KL}{\mathcal{KL}}
\newcommand{\cut}[1]{}
\newcommand{\trace}{\operatorname{trace}}

\newcommand{\bmu}{{\boldsymbol{\mu}}}
\newcommand{\btheta}{\boldsymbol{\theta}}
\newcommand{\bepsilon}{\boldsymbol{\epsilon}}
\newcommand{\balpha}{\boldsymbol{\alpha}}
\newcommand{\bbeta}{\boldsymbol{\beta}}
\newcommand{\bphi}{\boldsymbol{\phi}}
\newcommand{\bPhi}{\boldsymbol{\Phi}}
\newcommand{\bSigma}{\boldsymbol{\Sigma}}
\newcommand{\bpi}{\boldsymbol{\pi}}
\newcommand{\blambda}{\boldsymbol{\lambda}}

\newcommand{\argmax}{\operatorname{argmax}}
\newcommand{\argmin}{\operatorname{argmin}}
\newcommand{\ci}{{\bot\negthickspace\negthickspace\bot}} % conditional indep.
\newcommand{\neigh}{\operatorname{ne}}
\newcommand{\vectr}[2]{  \left[ \!\!\begin{array}{c} #1 \\
      #2 \end{array} \!\!\right]}
\newcommand{\deff}{\stackrel{\mathrm{def}}{=}}
\newcommand{\deldel}[2]{\frac{\partial #1}{\partial #2}}

\newcommand{\maketilde}{\raisebox{0.4ex}{\tiny $\sim$}}
\newcommand{\bfa}{\bm a}
\newcommand{\bfb}{\bm b}
\newcommand{\bfe}{\bm e}
\newcommand{\bff}{\bm f}
\newcommand{\bfk}{\bm k}
\newcommand{\bfm}{\bm m}
\newcommand{\bfn}{\bm n}
\newcommand{\bfp}{\bm p}
\newcommand{\bfs}{\bm s}
\newcommand{\bfu}{\bm u}
\newcommand{\bfx}{\bm x}
\newcommand{\bfy}{\bm y}
\newcommand{\bft}{\bm t}
\newcommand{\bfv}{\bm v}
\newcommand{\bfw}{\bm w}
\newcommand{\bfz}{\bm z}
\newcommand{\bfA}{\bm A}
\newcommand{\bfI}{\bm I}
\newcommand{\bfK}{\bm K}


\title{Gibbs Sampling}
\author{Carl Edward Rasmussen}
\date{October 28th, 2016}

\begin{document}


\begin{frame}
\titlepage
\end{frame}


\begin{frame}
\frametitle{Key concepts}

\begin{itemize}
\item \emph{inference} requires integrating out variables
\item Why may random sampling be useful for integration?
\item What happens if the joint distribution is too complicated to sample from?
\item Gibbs sampling and conditional distributions
\end{itemize}

\end{frame}


\begin{frame}
\frametitle{How do we do integrals wrt an intractable posterior?}

Approximate \Blue{expectations} of a function $\phi(\bfx)$ wrt
\Blue{probability} $p(\bfx)$:
\[
\E_{p(x)}[\phi(x)]\;=\;\bar\phi\;=\;\int \phi(\bfx)\Red{p(\bfx)}d\bfx,
\text{\ \ where\ \ }\bfx\in\R^D,
\]
when these are not analytically tractable, and typically $D\gg1$.
\begin{center}
\includegraphics[width=0.7\textwidth]{mc0}
\end{center}
Assume that we can evaluate $\phi(x)$ and $\Red{p(x)}$.
\end{frame}


\begin{frame}
\frametitle{Numerical integration on a grid}

Approximate the integral by a sum of products
\[
\int \phi(\bfx)\Red{p(\bfx)}d\bfx\;\simeq\;\sum_{\tau=1}^T\phi(\bfx^{(\tau)})\Red{p(\bfx^{(\tau)})}\Delta\bfx,
\]
where the $\bfx^{(\tau)}$ lie on an equidistant grid (or fancier
versions of this).

\centerline{\includegraphics[width=0.7\textwidth]{mc1}}

\Blue{Problem:} the number of grid points required, $k^D$, grows
exponentially with the dimension $D$. Practicable only to $D=4$ or so.
\end{frame}


\begin{frame}
\frametitle{Monte Carlo}

The fundamental basis for Monte Carlo approximations is
\[
\E_{\Red{p(x)}}[\phi(\bfx)]\;\simeq\;\hat\phi\;=\;\frac{1}{T}\sum_{\tau=1}^T\phi(\bfx^{(\tau)}),
\text{\ \ where\ \ }\bfx^{(\tau)}\sim \Red{p(\bfx)}.
\]

\centerline{\includegraphics[width=0.7\textwidth]{mc4}}

Under mild conditions, $\hat\phi\rightarrow\E[\phi(\bfx)]$ as
$T\rightarrow\infty$. For moderate $T$, $\hat\phi$ may still be a good
approximation. In fact it is an \emph{unbiased} estimate with
\[
\V[\hat\phi]\;=\;\frac{\V[\phi]}{T}, \text{\ \ where\ \ }
\V[\phi]\;=\;\int \big(\phi(\bfx)-\bar\phi\big)^2\Red{p(\bfx)}d\bfx.
\]
\Blue{Note}, that this variance is \Blue{\emph{independent}} of the dimension
$D$ of $\bfx$.
\end{frame}


\begin{frame}
\frametitle{Markov Chain Monte Carlo}

This is great, but \Blue{how do we generate random samples} from
$\Red{p(\bfx)}$?\\[1ex]

If $\Red{p(\bfx)}$ has a standard form, we may be
able to generate \Blue{\emph{independent}} samples.\\[1ex]

\underline{Idea:} could we design a Markov Chain,
$\Blue{q(\bfx'|\bfx)}$, which generates (dependent) samples from the
desired distribution $\Red{p(\bfx)}$?
\[
\bfx \rightarrow \bfx' \rightarrow \bfx'' \rightarrow  \bfx'''
\rightarrow \ldots
\]


One such algorithm is called \Blue{\emph{Gibbs sampling}}: for each
component $i$ of $\bfx$ in turn, sample a new value from the
conditional distribution of $x_i$ given all other variables:
\[
x_i'\;\sim\;p(x_i|x_1,\ldots,x_{i-1},x_{i+1},\ldots,x_D).
\]
It can be shown, that this will eventually generate dependent samples from the 
joint distribution $\Red{p(\bfx)}$.\\[1ex]

Gibbs sampling reduces the task of sampling from a joint distribution,
to sampling from a sequence of univariate conditional distributions.
\end{frame}


\begin{frame}
\frametitle{Gibbs sampling example: Multivariate Gaussian}

20 iterations of Gibbs sampling on a bivariate Gaussian; both
conditional distributions are Gaussian.

\begin{center}
\includegraphics[width=0.4\textwidth]{bvg-gibbs}
\end{center}

Notice that \Blue{strong correlations} can \Blue{slow down} Gibbs sampling.
\end{frame}


\begin{frame}
\frametitle{Gibbs Sampling}

Gibbs sampling is a parameter free algorithm, applicable if we know
how to sample from the conditional distributions.\\[1ex]

{\bf Main disadvantage:} depending on the target distribution, there may be
very strong correlations between consecutive samples.\\[1ex]

To get less dependence, Gibbs sampling is often run for a long time,
and the samples are thinned by keeping only every 10th or 100th
sample.\\[1ex]

Burn-in: often, the initial sequence of samples is discarded, until the chain has converged to the desired distribution. \Blue{What does \emph{convergence}} mean in this context?\\[1ex]

It is often challenging to judge the \emph{effective correlation
  length} of a Gibbs sampler. Sometimes several Gibbs samplers are run
from different starting points, to compare results.\\[1ex]


\end{frame}


\end{document}
