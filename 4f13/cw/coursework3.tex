\documentclass[12pt]{article}
\usepackage{fontspec} % This line only for XeLaTeX and LuaLaTeX
\setmainfont{Sabon LT Std} % text font Sabon
\usepackage[OT1,euler-digits]{eulervm} % math font euler
\usepackage{textcomp}
\usepackage{amssymb}
\usepackage{amsmath}

\parskip=2ex
\parindent=0ex
\textheight=270mm
\textwidth=7in
\oddsidemargin=-0.25in
\topmargin=-25mm
\pagestyle{empty}

\pagestyle{empty}
\newcommand{\bi}{\begin{itemize}\vspace*{-0.3in}}
\newcommand{\ei}{\end{itemize}\vspace*{-0.3in}}
\newcommand{\im}{\item\vspace*{-0.4in}}
\newcommand{\new}{^{\scriptscriptstyle \mathrm{new}}}
\newcommand{\bea}{\begin{eqnarray*}}
\newcommand{\eea}{\end{eqnarray*}}
\newcommand{\T}{{\scriptsize^{\top}}}
\newcommand{\yy}{{\mathbf y}}
\newcommand{\by}{{\mathbf y}}
\newcommand{\xx}{{\mathbf x}}
\newcommand{\bx}{{\mathbf x}}
\newcommand{\zz}{{\mathbf z}}
\newcommand{\Bet}{{\rm Beta}}
\newcommand{\deldel}[2]{\frac{\partial #1}{\partial #2}}
\newcommand{\hh}{\hspace{0.2in}}
\newcommand{\bpi}{{\boldsymbol{\pi}}}
\newcommand{\bmu}{{\boldsymbol{\mu}}}
\newcommand{\btheta}{{\boldsymbol{\theta}}} % amssymb
\newcommand{\blambda}{{\boldsymbol{\lambda}}} % amssymb
\newcommand{\bs}{{\mathbf s}}
\newcommand{\be}{\begin{equation*}}
\newcommand{\ee}{\end{equation*}}

\begin{document}

\centerline{\large\bf 4F13 Machine Learning: Coursework \#3: Latent
  Dirichlet Allocation}

\vspace{0.3cm}

\centerline{Carl Edward Rasmussen \& Hong Ge} 

\vspace{0.2cm}

\centerline{Due (for non-MLMI students): 12:00 noon, Friday Dec 6th, 2024 online via moodle}

\vspace{2mm}

Your answers should contain an explanation of what you do, and 2-4
central commands to achieve it (but complete listings are unnecessary
amd discouraged). You must also give an \emph{interpretation} of what
the numerical values and graphs you provide \emph{mean} -- why are the
results the way they are? {\bf Each question should be labelled and
  answered separately and distinctly.} Total combined length of
answers must not exceed 5 sides of a4 (plus cover page), minimum 11pt
font, 1 inch margins. All questions carry approximately equal weight.

In this assignment, we will give you two short pieces of matlab or python code,
which implement the main ingredients of Gibbs sampling for a Mixture
of Multinomials: \texttt{bmm}, and for LDA: \texttt{lda}. Before you
start answering questions, you should spend some time understanding in
detail, what this code does. This will enable you to answer all the
questions with very little programming effort on your part.

The data is in the file \texttt{kos\_doc\_data.mat}. The word counts
are in the matrix variables \texttt{A} and \texttt{B} for training and
testing respectively, both matrices with 3 columns: document ID, word
ID and word count. The words themselves are the variable \texttt{V}. 
%
\begin{enumerate}

\item[a)] Using the training data in \texttt{A}, find the maximum
  likelihood multinomial over words, and show the 20 largest
  probability items in a histogram. You may use the \texttt{barh}
  command. For that multinomial model, what is the highest and lowest
  possible test set log probability (for any possible test set)?
  Explain the implications of this.

\item[b)] Instead of the maximum likelihood fit in question a), do
  Bayesian inference using a symmetric Dirichlet prior with a
  concentration parameter $\alpha$ on the word probabilities.  Compare
  the expressions for the predictive word probabilities for these two
  types of inference, and explain the implications, both for common
  and rare words for both small and large values of $\alpha$.

\item[c)] For the Bayesian model, what is the log probability for the
  test document with ID 2001? Explain whether, when computing the log
  probability of a test document, you would use the multinomial or the
  categorical distribution function? What is the per-word perplexity
  for the document with ID 2001? What is the per-word perplexity over
  all documents in \texttt{B}? Explain why the perplexities are
  different for different documents? What would the perplexity be for
  a uniform multinomial?

\item[d)] The \texttt{bmm} script implements Gibbs sampling for a
  mixture of multinomials model. Use and modify the script to plot the
  evolution of the mixing proportions as a function of the number of
  Gibbs sweeps. Give a detailed account of whether and when the Gibbs
  sampler converges to the posterior distribution.

\item[e)] Use and modify \texttt{lda}. Plot topic posteriors for
  $K=20$ as a function of the number of Gibbs sweeps, up to 50
  sweeps. Comment on these. Compute the perplexity for the documents
  in \texttt{B} for the state after 50 Gibbs sweeps, and compare to
  previously computed perplexities. Are 50 Gibbs sweeps adequate?
  Plot the word entropy (what units do you use?) for each of the
  topics and explain the evolution as a function of the number of
  Gibbs sweeps. Explain whether and why you think $K=20$ is appropriate for
  this data set.
\end{enumerate}

\end{document}
