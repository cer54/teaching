\documentclass[10pt]{beamer}
\usefonttheme{professionalfonts,serif}
\def\newblock{\hskip .11em plus .33em minus .07em}
\usepackage[numbers,sort]{natbib}
\renewcommand{\rmdefault}{psbx}
\usepackage[utf8]{inputenc}
\usepackage[T1]{fontenc}
\usepackage{textcomp}
\usepackage{eulervm}

\usetheme{default}           % tips from David Blei
\useinnertheme{circles}
\useoutertheme{infolines}
\setbeamertemplate{headline}{}
\setbeamertemplate{navigation symbols}{}
\setbeamerfont{itemize/enumerate subbody}{size=\normalsize}
\setbeamerfont{itemize/enumerate subsubbody}{size=\normalsize}
\usecolortheme{seahorse}
\setbeamersize{text margin left=2mm,text margin right=2mm}

\graphicspath{{../../figures/}}

\definecolor{mypine}{rgb}{0.05,0.45,0.05}
\definecolor{mycyan}{rgb}{0.0,0.9,0.9}
\newcommand{\Red}{\textcolor{red}}
\newcommand{\Blue}{\textcolor{blue}}
\newcommand{\Green}{\textcolor{mypine}}
\newcommand{\PineGreen}{\textcolor{mypine}}
\newcommand{\Magenta}{\textcolor{magenta}}
\newcommand{\Cyan}{\textcolor{mycyan}}

\newcommand{\N}{\mathcal{N}}
\newcommand{\R}{\mathbb{R}}
\newcommand{\T}{{\scriptsize^{\top}}}
\newcommand{\D}{\mathcal{D}}
\newcommand{\F}{\mathcal{F}}
\newcommand{\E}{\mathbb{E}}
\newcommand{\V}{\mathbb{V}}
\newcommand{\M}{\mathcal{M}}
\newcommand{\KL}{\mathcal{KL}}
\newcommand{\cut}[1]{}
\newcommand{\trace}{\operatorname{trace}}

\newcommand{\bmu}{{\boldsymbol{\mu}}}
\newcommand{\btheta}{\boldsymbol{\theta}}
\newcommand{\bepsilon}{\boldsymbol{\epsilon}}
\newcommand{\balpha}{\boldsymbol{\alpha}}
\newcommand{\bbeta}{\boldsymbol{\beta}}
\newcommand{\bphi}{\boldsymbol{\phi}}
\newcommand{\bPhi}{\boldsymbol{\Phi}}
\newcommand{\bSigma}{\boldsymbol{\Sigma}}
\newcommand{\bpi}{\boldsymbol{\pi}}
\newcommand{\blambda}{\boldsymbol{\lambda}}

\newcommand{\argmax}{\operatorname{argmax}}
\newcommand{\argmin}{\operatorname{argmin}}
\newcommand{\ci}{{\bot\negthickspace\negthickspace\bot}} % conditional indep.
\newcommand{\neigh}{\operatorname{ne}}
\newcommand{\vectr}[2]{  \left[ \!\!\begin{array}{c} #1 \\
      #2 \end{array} \!\!\right]}
\newcommand{\deff}{\stackrel{\mathrm{def}}{=}}
\newcommand{\deldel}[2]{\frac{\partial #1}{\partial #2}}

\newcommand{\maketilde}{\raisebox{0.4ex}{\tiny $\sim$}}
\newcommand{\bfa}{\mathbf a}
\newcommand{\bfb}{\mathbf b}
\newcommand{\bfe}{\mathbf e}
\newcommand{\bff}{\mathbf f}
\newcommand{\bfk}{\mathbf k}
\newcommand{\bfm}{\mathbf m}
\newcommand{\bfn}{\mathbf n}
\newcommand{\bfp}{\mathbf{p}}
\newcommand{\bfs}{\mathbf s}
\newcommand{\bfu}{\mathbf u}
\newcommand{\bfx}{\mathbf x}
\newcommand{\bfy}{\mathbf y}
\newcommand{\bft}{\mathbf t}
\newcommand{\bfv}{\mathbf v}
\newcommand{\bfw}{\mathbf w}
\newcommand{\bfA}{\mathbf A}
\newcommand{\bfI}{\mathbf I}
\newcommand{\bfK}{\mathbf K}


\title{Probability fundamentals}
\author{Carl Edward Rasmussen}
\date{June 29th, 2016}

\begin{document}

\begin{frame}
\titlepage
\end{frame}

%%%%%%%%%%%%%%%%%%%%%%%%%%%%%%%%%%%%%%%%%%%%%%%%%%%%%%%%%%%%%%%%%%%%%%
\begin{frame}
\frametitle{Key Concepts}

\begin{itemize}
\item probability basics
\begin{itemize}
\item Example: Medical diagnosis
\item joint, conditional and marginal probabilities
\item the two rules of probability: sum and product rules
\item Bayes rule
\end{itemize}
\item Bayesian inference and prediction with finite regression models
\begin{itemize}
\item likelihood and prior
\item posterior and predictive distribution
\end{itemize}
\item the marginal likelihood
\begin{itemize}
\item Bayesian model selection
\item Example: How Bayes avoids overfitting
\end{itemize}
\end{itemize}
\end{frame}


\begin{frame}
\frametitle{Medical inference (diagnosis)}

Breast cancer facts:
\begin{itemize}
\item $1\%$ of scanned women have breast cancer
\item $80\%$ of women with breast cancer get positive mammography scans
\item $9.6\%$ of women without breast cancer also get positive
  mammography scans
\end{itemize}

{\bf Question:} A woman gets a scan, and it is positive; what is the probability that
she has breast cancer?

\begin{enumerate}
\item less than $1\%$
\item around $10\%$
\item around $90\%$
\item more than $99\%$
\end{enumerate}

\end{frame}
%%%%%%%%%%%%%%%%%%%%%%%%%%%%%%%%%%%%%%%%%%%%%%%%%%%%%%%%%%%%%%%%%%%%%%
\begin{frame}
\frametitle{Medical inference}

Breast cancer facts:
\begin{itemize}
\item $1\%$ of scanned women have breast cancer
\item $80\%$ of women with breast cancer get positive mammography scans
\item $9.6\%$ of women without breast cancer also get positive
  mammography scans
\end{itemize}

Define: $C$ = presence of breast cancer; $\bar{C}$ = no breast cancer. \\
$M$ = scan is positive; $\bar{M}$ = scan is negative.

The probability of cancer for scanned women is $p(C) = 1\%$

If there is cancer, the probability of a positive mammography
is $p(M|C) = 80\%$

If there is no cancer, we still have $p(M|\bar{C}) = 9.6\%$

The question is what is $p(C|M)$?

\end{frame}
%%%%%%%%%%%%%%%%%%%%%%%%%%%%%%%%%%%%%%%%%%%%%%%%%%%%%%%%%%%%%%%%%%%%%%
\begin{frame}
\frametitle{Medical inference}

What is $p(C|M)$?

Consider 10000 subjects of screening
\begin{itemize}
\item $p(C) = 1\%$, therefore 100 of them have cancer, of which
\begin{itemize}
\item $p(M|C) = 80\%$, therefore 80 get a positive mammography
\item 20 get a negative mammography
\end{itemize}
\item $p(\bar{C}) = 99\%$, therefore 9900 of them do not have cancer, of which
\begin{itemize}
\item $p(M|\bar{C}) = 9.6\%$, therefore 950 get a positive mammography
\item 8950 get a  negative mammography
\end{itemize}
\end{itemize}

\begin{center}
\begin{tabular}[h]{c|c|c}
& $M$ & $\bar M$\\\hline
\rule{0mm}{5mm}$C$ & \Red{80} & 20\\\hline
\rule{0mm}{5mm}$\bar C$ & \Blue{950} & 8950 \\\hline
\end{tabular}
\end{center}


\end{frame}
%%%%%%%%%%%%%%%%%%%%%%%%%%%%%%%%%%%%%%%%%%%%%%%%%%%%%%%%%%%%%%%%%%%%%%
\begin{frame}

What is $p(C|M)$?

\begin{center}
\begin{tabular}[h]{c|c|c}
& $M$ & $\bar M$\\\hline
\rule{0mm}{5mm}$C$ & \Red{80} & 20\\\hline
\rule{0mm}{5mm}$\bar C$ & \Blue{950} & 8950 \\\hline
\end{tabular}
\end{center}

$p(C|M)$ is obtained as the proportion of all positive
mammographies for which there actually is breast cancer

\[
p(C|M)\;=\;\frac{p(C,M)}{p(C,M) + p(\bar C ,M)}
\;=\;\frac{p(C,M)}{p(M)}\;=\;\frac{\Red{80}}{\Red{80} + \Blue{950}}\;\simeq\;7.8\%
\]

This is an example of Bayes' rule:
\[
p(A|B) \;=\; \frac{p(B|A)p(A)}{p(B)}.
\]
Which is just a consequence of the definition of \Blue{\emph{conditional
  probability}}
\[
p(A|B)\;=\;\frac{p(A,B)}{p(B)},\text{\ \ \ \ (where $p(B)\neq 0$)}.
\]
\end{frame}
%%%%%%%%%%%%%%%%%%%%%%%%%%%%%%%%%%%%%%%%%%%%%%%%%%%%%%%%%%%%%%%%%%%%%%
\begin{frame}
\frametitle{Just two rules of probability theory}

Astonishingly, the rich theory of probability can be derived using just two rules:

The \Red{\emph{sum rule}} states that
\[
p(A)\;=\;\sum_B p(A,B), \text{\ \ \ \ or\ \ \ \ } 
p(A)\;=\;\int_B p(A,B)dB,
\]
for discrete and continuous variables. Sometimes called \Blue{\emph{marginalization}}.

The \Red{\emph{product rule}} states that
\[
p(A,B)\;=\;p(A|B) p(B).
\]
It follows directly from the definition of \Blue{conditional
  probability}, and leads directly to \Blue{Bayes' rule}
\[
\Blue{p(A|B)p(B)}\;=\;p(A,B)\;=\;\Blue{p(B|A)p(A)}\;\;\Rightarrow\;\;
p(A|B)\;=\;\frac{p(B|A)p(A)}{p(B)}.
\]
Special case:\\[0.2ex] if $A$ and $B$ are \Blue{\emph{independent}},
$p(A|B)\;=\;p(A)$, and thus $p(A,B)\;=\;p(A)p(B)$.
\end{frame}

\end{document}
