\documentclass[10pt]{beamer}
\usefonttheme{professionalfonts,serif}
\def\newblock{\hskip .11em plus .33em minus .07em}
\usepackage[numbers,sort]{natbib}

\usepackage{fontspec} % This line only for XeLaTeX and LuaLaTeX
\setmainfont{Sabon LT Std} % text font Sabon
\usepackage[OT1,euler-digits]{eulervm} % math font euler

%\renewcommand{\rmdefault}{psbx}
%\usepackage[utf8]{inputenc}
%\usepackage[T1]{fontenc}
\usepackage{textcomp}
%\usepackage{eulervm}
\usepackage{bm}

\usetheme{default}           % tips from David Blei
\useinnertheme{circles}
\useoutertheme{infolines}
\setbeamertemplate{headline}{}
\setbeamertemplate{navigation symbols}{}
\setbeamerfont{itemize/enumerate subbody}{size=\normalsize}
\setbeamerfont{itemize/enumerate subsubbody}{size=\normalsize}
\usecolortheme{seahorse}
\setbeamersize{text margin left=2mm,text margin right=2mm}

\graphicspath{{../../figures/}}

\definecolor{mypine}{rgb}{0.05,0.45,0.05}
\definecolor{mycyan}{rgb}{0.0,0.9,0.9}
\newcommand{\Red}{\textcolor{red}}
\newcommand{\Blue}{\textcolor{blue}}
\newcommand{\Green}{\textcolor{mypine}}
\newcommand{\PineGreen}{\textcolor{mypine}}
\newcommand{\Magenta}{\textcolor{magenta}}
\newcommand{\Cyan}{\textcolor{mycyan}}

\newcommand{\N}{\mathcal{N}}
\newcommand{\R}{\mathbb{R}}
\newcommand{\T}{{\scriptsize^{\top}}}
\newcommand{\D}{\mathcal{D}}
\newcommand{\F}{\mathcal{F}}
\newcommand{\E}{\mathbb{E}}
\newcommand{\V}{\mathbb{V}}
\newcommand{\M}{\mathcal{M}}
\newcommand{\KL}{\mathcal{KL}}
\newcommand{\cut}[1]{}
\newcommand{\trace}{\operatorname{trace}}

\newcommand{\bmu}{{\boldsymbol{\mu}}}
\newcommand{\btheta}{\boldsymbol{\theta}}
\newcommand{\bepsilon}{\boldsymbol{\epsilon}}
\newcommand{\balpha}{\boldsymbol{\alpha}}
\newcommand{\bbeta}{\boldsymbol{\beta}}
\newcommand{\bphi}{\boldsymbol{\phi}}
\newcommand{\bPhi}{\boldsymbol{\Phi}}
\newcommand{\bSigma}{\boldsymbol{\Sigma}}
\newcommand{\bpi}{\boldsymbol{\pi}}
\newcommand{\blambda}{\boldsymbol{\lambda}}

\newcommand{\argmax}{\operatorname{argmax}}
\newcommand{\argmin}{\operatorname{argmin}}
\newcommand{\ci}{{\bot\negthickspace\negthickspace\bot}} % conditional indep.
\newcommand{\neigh}{\operatorname{ne}}
\newcommand{\vectr}[2]{  \left[ \!\!\begin{array}{c} #1 \\
      #2 \end{array} \!\!\right]}
\newcommand{\deff}{\stackrel{\mathrm{def}}{=}}
\newcommand{\deldel}[2]{\frac{\partial #1}{\partial #2}}

\newcommand{\maketilde}{\raisebox{0.4ex}{\tiny $\sim$}}
\newcommand{\bfa}{\bm a}
\newcommand{\bfb}{\bm b}
\newcommand{\bfe}{\bm e}
\newcommand{\bff}{\bm f}
\newcommand{\bfk}{\bm k}
\newcommand{\bfm}{\bm m}
\newcommand{\bfn}{\bm n}
\newcommand{\bfp}{\bm p}
\newcommand{\bfs}{\bm s}
\newcommand{\bfu}{\bm u}
\newcommand{\bfx}{\bm x}
\newcommand{\bfy}{\bm y}
\newcommand{\bft}{\bm t}
\newcommand{\bfv}{\bm v}
\newcommand{\bfw}{\bm w}
\newcommand{\bfz}{\bm z}
\newcommand{\bfA}{\bm A}
\newcommand{\bfI}{\bm I}
\newcommand{\bfK}{\bm K}


\title{Bayesian inference and prediction with finite regression models}
\author{Carl Edward Rasmussen}
\date{July 1st, 2016}

\begin{document}

\begin{frame}
\titlepage
\end{frame}

\begin{frame}
\frametitle{Key concepts}

\end{frame}


\begin{frame}
\frametitle{Posterior probability of a function}

Given the \Blue{prior}  functions \Blue{$p(\bff)$} how can we make predictions?
\begin{itemize}
\item Of all functions generated from the prior, keep those that fit the data.
\item The notion of closeness to the data is given by the \Red{likelihood $p(\bf
y|\bff)$}.
\item We are really interested in the posterior distribution over functions:
%
\[
\Green{p(\bff|\bfy)}\;=\;\frac{\Red{p(\bfy|\bff)}\,\Blue{p(\bff)}}{\Cyan{p(\bfy)
}}
{\; \; \; \; \; \;} \mathrm{Bayes\; Rule}
\]
%
\end{itemize}

\parbox{0.45\textwidth}{
\centerline{\includegraphics[width=0.45\textwidth]{random_polynomials_degree17.pdf}}
\centerline{Some samples from the prior}
}
\parbox{0.45\textwidth}{
\centerline{\includegraphics[width=0.45\textwidth]{samples_posterior_degree17.pdf}}
\centerline{Samples from the posterior}
}
\end{frame}


\begin{frame}
\frametitle{Priors on parameters induce priors on functions}

A model $\mathcal{M}$ is the choice of a \Red{model structure} and of \Blue{para
meter values}.  
%
\[
f_\bfw(x)\; =\; \sum_{m=0}^{\Red{M}} \Blue{w_m}\,\Red{\phi_m(x)}
\]
%
The prior $p(\bfw|\mathcal{M})$ determines what \Blue{functions} this model can generate. Example: \vspace*{-1.5ex}
\begin{itemize}
\item Imagine we choose $M=17$, and $p(w_m)=\N(w_m;\;0, \sigma_\bfw^2)$.
\item We have actually defined a \Blue{prior distribution over functions $p(\bff|\mathcal{M})$}.
\end{itemize} \vspace*{-1.5ex}

\parbox{0.49\textwidth}{
This figure is generated as follows:
\begin{itemize}
\item Use polynomial basis functions, $\phi_m(x)=x^m$.
\item Define a uniform grid of $n=100$ values in $x$ from $[-1.5, 2]$.
\item Generate matrix $\bPhi$ for $M=17$.
%\texttt{Phi = repmat(x,1,M+1).\^{}repmat(0:M,N,1);}
\item Draw $w_m\sim{\cal N}(0,1)$.
\item Compute and plot $\bff=\bPhi_{n\times 18}\,\bfw$.
\end{itemize}
}
\parbox{0.45\textwidth}{
\centerline{\includegraphics[width=0.5\textwidth]{random_polynomials_degree17.pdf}}
}

\end{frame}


\cut{\begin{frame}
\frametitle{Priors over functions from other finite linear models}

We could have used very different types of basis functions. Two examples:

\parbox{0.49\textwidth}{
\begin{itemize}
\item Localised basis functions:
%
\[
\phi_m(x_n) = \exp\left(-(x_n-c_m)^2/\,\gamma^2\right)
\]
%
\end{itemize}
\centerline{\includegraphics[width=0.49\textwidth]{fig_rvm_prior_gaussbf.pdf}}
}
\parbox{0.49\textwidth}{
\begin{itemize}
\item
Alternative basis functions:
%
\[
\phi_m(x_n) = \log\left(1+(x_n-c_m)^2/\,\gamma^2\right)
\]
%
\end{itemize}
\centerline{\includegraphics[width=0.49\textwidth]{fig_rvm_prior_tanhbf.pdf}}
}

\vfill

Some remarks:
\begin{itemize}
\item We are not really interested in the $w_m$, they are \Red{nuisance parameters}.
\item We want to concentrate on specifying the \Blue{prior over functions $p(\bff)$ directly}.
\end{itemize}

\end{frame}
}


\begin{frame}
\frametitle{Maximum likelihood, parametric model}

Supervised parametric learning:
\begin{itemize}
\item data: ${\bf x}, {\bf y}$
\item model $\mathcal{M}$: $y = f_{\bf w}(x) + \varepsilon$
\end{itemize}

Gaussian likelihood:
\[
\Red{p({\bf y}|{\bf x},{\bf w}, \mathcal{M})}\;\propto\;
\prod_{n=1}^N\exp(-\tfrac{1}{2}(y_n-f_{\bf w}(x_n))^2/\sigma^2_{\rm noise}).
\]

Maximize the likelihood:
\[
\Blue{{\bf w}_{\rm ML}}\;=\;\operatornamewithlimits{argmax}_{\bf w}
\Red{p({\bf y}|{\bf x},{\bf w},\mathcal{M})}.
\]

Make predictions, by plugging in the ML estimate:
\[
\Red{p(y_*|x_*,\Blue{{\bf w}_{\rm ML}},\mathcal{M})}
\]

\vfill

\end{frame}


\begin{frame}
\frametitle{Bayesian inference, parametric model, cont.}

Posterior parameter distribution by Bayes rule ({\small $\Green{p(a|b)}=\Blue{p(a)} \Red{p(b|a)}/\Cyan{p(b)}$}):
\[
\Green{p({\bf w}|{\bf x},{\bf y},\mathcal{M})}\;=\;\frac{\Blue{p({\bf w}|\mathcal{M})}
\Red{p({\bf y}|{\bf x},{\bf w},\mathcal{M})}}{\Cyan{p({\bf y}|{\bf x},\mathcal{M})}}
\]

Making predictions (marginalizing out the parameters):
\[
\begin{split}
p(y_*|x_*,{\bf x}, {\bf y},\mathcal{M})\;&=\;
\int p(y_*,{\bf w}|{\bf x}, {\bf y}, x_*, \mathcal{M})d{\bf w}\\
&=\;\int \Red{p(y_*|{\bf w},x_*,\mathcal{M})}\Green{p({\bf w}|{\bf x}, {\bf y},\mathcal{M})}d{\bf w}.
\end{split}
\]

\end{frame}


\begin{frame}
\frametitle{Posterior and predictive distribution in detail}

For a linear-in-the-parameters model with Gaussian priors and Gaussian noise:
\begin{itemize}
\item Gaussian \Blue{\emph{prior}} on the weights: 
$\Blue{p(\bfw|\mathcal{M})}=\N(\bfw;\;\mathbf{0},\,\sigma_\bfw^2\,\bfI)$
\item Gaussian \Red{\emph{likelihood}} of the weights:
$\Red{p(\bfy|\bfx,\bfw,\mathcal{M}})=\N(\bfy;\;\bPhi\,\bfw,\,\sigma_\mathrm{noise}^2\,\bfI)$
\end{itemize}

\Green{Posterior} parameter distribution by Bayes rule $p(a|b)=p(a)p(b|a)/p(b)$:
\[
\Green{p({\bf w}|{\bf x},{\bf y},\mathcal{M})}\;=\;\frac{\Blue{p({\bf w}|\mathcal{M})}
\Red{p({\bf y}|{\bf x},{\bf w},\mathcal{M})}}{\Cyan{p({\bf y}|{\bf x},\mathcal{M})}}
\;=\; \N(\bfw;\;\bmu,\,\bSigma)
\]
\[
\bSigma\;=\;\left(\sigma_\mathrm{noise}^{-2}\bPhi^\top\bPhi+\sigma_\bfw^{-2}\,\bfI\right)^{-1}
\mathrm{\; \; \; and \; \; \; \; }
\bmu\;=\;
\Big(\bPhi^\top\bPhi+\frac{\sigma_\mathrm{noise}^2}{\sigma_\bfw^2}\,\bfI\Big)^{-1}\bPhi^\top\bfy
\]

The predictive distribution is given by:
\[
p(y_*|x_*,{\bf x}, {\bf y},\mathcal{M})\;=\;
\N(y_*;\; \bphi(x_*)^\top\bmu,\,\bphi(x_*)^\top\bSigma\bphi(x_*)+\sigma_\mathrm{noise}^2)
\]

\end{frame}

\end{document}
