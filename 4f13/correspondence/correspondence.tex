\documentclass[10pt]{beamer}
\usefonttheme{professionalfonts,serif}
\def\newblock{\hskip .11em plus .33em minus .07em}
\usepackage[numbers,sort]{natbib}
\renewcommand{\rmdefault}{psbx}
\usepackage[utf8]{inputenc}
\usepackage[T1]{fontenc}
\usepackage{textcomp}
\usepackage{eulervm}

\usetheme{default}           % tips from David Blei
\useinnertheme{circles}
\useoutertheme{infolines}
\setbeamertemplate{headline}{}
\setbeamertemplate{navigation symbols}{}
\setbeamerfont{itemize/enumerate subbody}{size=\normalsize}
\setbeamerfont{itemize/enumerate subsubbody}{size=\normalsize}
\usecolortheme{seahorse}
\setbeamersize{text margin left=2mm,text margin right=2mm}

\graphicspath{{../../figures/}}

\definecolor{mypine}{rgb}{0.05,0.45,0.05}
\definecolor{mycyan}{rgb}{0.0,0.9,0.9}
\newcommand{\Red}{\textcolor{red}}
\newcommand{\Blue}{\textcolor{blue}}
\newcommand{\Green}{\textcolor{mypine}}
\newcommand{\PineGreen}{\textcolor{mypine}}
\newcommand{\Magenta}{\textcolor{magenta}}
\newcommand{\Cyan}{\textcolor{mycyan}}

\newcommand{\N}{\mathcal{N}}
\newcommand{\R}{\mathbb{R}}
\newcommand{\T}{{\scriptsize^{\top}}}
\newcommand{\D}{\mathcal{D}}
\newcommand{\F}{\mathcal{F}}
\newcommand{\E}{\mathbb{E}}
\newcommand{\V}{\mathbb{V}}
\newcommand{\M}{\mathcal{M}}
\newcommand{\KL}{\mathcal{KL}}
\newcommand{\cut}[1]{}
\newcommand{\trace}{\operatorname{trace}}

\newcommand{\bmu}{{\boldsymbol{\mu}}}
\newcommand{\btheta}{\boldsymbol{\theta}}
\newcommand{\bepsilon}{\boldsymbol{\epsilon}}
\newcommand{\balpha}{\boldsymbol{\alpha}}
\newcommand{\bbeta}{\boldsymbol{\beta}}
\newcommand{\bphi}{\boldsymbol{\phi}}
\newcommand{\bPhi}{\boldsymbol{\Phi}}
\newcommand{\bSigma}{\boldsymbol{\Sigma}}
\newcommand{\bpi}{\boldsymbol{\pi}}
\newcommand{\blambda}{\boldsymbol{\lambda}}

\newcommand{\argmax}{\operatorname{argmax}}
\newcommand{\argmin}{\operatorname{argmin}}
\newcommand{\ci}{{\bot\negthickspace\negthickspace\bot}} % conditional indep.
\newcommand{\neigh}{\operatorname{ne}}
\newcommand{\vectr}[2]{  \left[ \!\!\begin{array}{c} #1 \\
      #2 \end{array} \!\!\right]}
\newcommand{\deff}{\stackrel{\mathrm{def}}{=}}
\newcommand{\deldel}[2]{\frac{\partial #1}{\partial #2}}

\newcommand{\maketilde}{\raisebox{0.4ex}{\tiny $\sim$}}
\newcommand{\bfa}{\mathbf a}
\newcommand{\bfb}{\mathbf b}
\newcommand{\bfe}{\mathbf e}
\newcommand{\bff}{\mathbf f}
\newcommand{\bfk}{\mathbf k}
\newcommand{\bfm}{\mathbf m}
\newcommand{\bfn}{\mathbf n}
\newcommand{\bfp}{\mathbf{p}}
\newcommand{\bfs}{\mathbf s}
\newcommand{\bfu}{\mathbf u}
\newcommand{\bfx}{\mathbf x}
\newcommand{\bfy}{\mathbf y}
\newcommand{\bft}{\mathbf t}
\newcommand{\bfv}{\mathbf v}
\newcommand{\bfw}{\mathbf w}
\newcommand{\bfA}{\mathbf A}
\newcommand{\bfI}{\mathbf I}
\newcommand{\bfK}{\mathbf K}


\title{Linear in the parameters models and GP}
\author{Carl Edward Rasmussen}
\date{October 13th, 2016}

\begin{document}

\begin{frame}
\titlepage
\end{frame}

\begin{frame}
\frametitle{Key concepts}
\begin{itemize}
\item We give an interpretation of the marginal likelihood in terms of
\begin{itemize}
\item a data fit
\item a complexity penalty
\end{itemize}
\item covariance functions can be parameterized using hyperparameters
\item hyperparameters can be fit by optimizing the marginal likelihood
\begin{itemize}
\item this is a form of model selection
\end{itemize}
\item Occam's razor is automatic and avoids overfitting
\end{itemize}
\end{frame}

\begin{frame}
\frametitle{From random functions to covariance functions}

Consider the class of linear functions:
\[
f(x)\;=\;ax+b,\text{\ \ where\ \ }a\sim{\cal N}(0,\alpha),\text{\ \ and\ \ }
b\sim{\cal N}(0,\beta).
\]
We can compute the mean function:
\[
\mu(x)\;=\;E[f(x)]\;=\;\int\!\int f(x)p(a)p(b)dadb\;=\;
\int axp(a)da+\int bp(b)db=0,
\]
and covariance function:
\[
\begin{split}
k(x,x')\;=&\;E[(f(x)-0)(f(x')-0)]\;=\;\int\!\int
(ax+b)(ax'+b)p(a)p(b)dadb\\
=&\;\int a^2xx'p(a)da+\int b^2p(b)db+(x+x')\int abp(a)p(b)dadb
=\alpha xx'+\beta.
\end{split}
\]
\end{frame}


\begin{frame}
\frametitle{From finite linear models to Gaussian processes (1)}
Finite linear model with Gaussian priors on the weights:
%
\[
f(x)\;=\;\sum_{m=1}^M w_m\,\phi_m(x) {\hspace{2cm}} 
p(\bfw)=\N(\bfw;\; \mathbf{0}, A)
\]
%
The joint distribution of any $\bff=[f(x_1),\ldots,f(x_N)]^\top$ is a
multivariate Gaussian -- this looks like a Gaussian Process!

The prior $p(\bff)$ is fully characterized by the \Blue{\it mean} and \Red{\it covariance} 
functions.
%
\begin{align*}
\Blue{m(x)\;=\;E_\bfw\big(f(x)\big)}
&\;=\;\int\Big(\sum_{m=1}^M w_k\phi_m(x)\Big) p(\bfw) d\bfw
\;=\;\sum_{m=1}^M \phi_m(x)\int w_m p(\bfw) d \bfw\\
&\;=\;\sum_{m=1}^M \phi_m(x)\int w_m p(w_m) dw_m\;=\;0
\end{align*}
The \Blue{\emph{mean function}} is zero.
 
\end{frame}


\begin{frame}
\frametitle{From finite linear models to Gaussian processes (2)}

\Red{Covariance function} of a finite linear model
%
\[
\begin{array}{l}
f(x)\;=\;\sum_{m=1}^M w_m\,\phi_m(x)\;=\;\bfw^\top\bphi(x)\\[1mm]
p(\bfw)\;=\;\N(\bfw;\; \mathbf{0}, A)
\end{array}
{\; \; \; \; \; }
\begin{array}{lr}
\bphi(x) = [\phi_1(x),\ldots,\phi_M(x)]^\top \!\!\! & \!\!\! _{(M \times 1)}
\end{array}
\]
%
%
\begin{align*}
&\Red{k(x_i, x_j)=Cov_\bfw \big(f(x_i), f(x_j)\big)}=E_\bfw\big(f(x_i)f(x_j)\big)
 -\underbrace{E_\bfw\big(f(x_i)\big)E_\bfw\big(f(x_j)\big)}_{0}\\[-3mm]
&=\int\!...\!\int\Big(\sum_{k=1}^M\sum_{l=1}^M w_kw_l\phi_k(x_i)\phi_l(x_j)\Big)
 p(\bfw)\,d\bfw\\
&=\sum_{k=1}^M\sum_{l=1}^M \phi_k(x_i)\phi_l(x_j) 
\underbrace{\iint w_kw_l p(w_k, w_l) dw_kdw_l}_{A_{kl}}
= \sum_{k=1}^M\sum_{l=1}^M A_{kl}\phi_k(x_i)\phi_l(x_j)
\end{align*}
%
\[
\boxed{\Red{k(x_i,x_j)}=\bphi(x_i)^\top A \bphi(x_j)}
\]
%

Note: If $A=\sigma_\bfw^2 \bfI$ then 
$k(x_i,x_j)=\sigma_\bfw^2 \sum_{k=1}^M \phi_k(x_i)\phi_k(x_j)=\sigma_\bfw^2\bphi(x_i)^\top\bphi(x_j)$

\end{frame}


\begin{frame}
\frametitle{GPs and Linear in the parameters models are equivalent}

We've seen that a Linear in the parameters model, with a Gaussian prior on
the weights is also a GP.

Note the different computational complexity: GP: ${\cal O}(N^3)$,
linear model ${\cal O}(NM^2)$ where $M$ is the number of basis
functions and $N$ the number of training cases.

So, which representation is most efficient?

Might it also be the case that every GP corresponds to a Linear in the
parameters model?\hfill (Mercer's theorem.)
\end{frame}

\end{document}
