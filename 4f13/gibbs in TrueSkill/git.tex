\documentclass[10pt]{beamer}
\usefonttheme{professionalfonts,serif}
\def\newblock{\hskip .11em plus .33em minus .07em}
\usepackage[numbers,sort]{natbib}

\usepackage{fontspec} % This line only for XeLaTeX and LuaLaTeX
\setmainfont{Sabon LT Std} % text font Sabon
\usepackage[OT1,euler-digits]{eulervm} % math font euler

%\renewcommand{\rmdefault}{psbx}
%\usepackage[utf8]{inputenc}
%\usepackage[T1]{fontenc}
\usepackage{textcomp}
%\usepackage{eulervm}
\usepackage{bm}

\usetheme{default}           % tips from David Blei
\useinnertheme{circles}
\useoutertheme{infolines}
\setbeamertemplate{headline}{}
\setbeamertemplate{navigation symbols}{}
\setbeamerfont{itemize/enumerate subbody}{size=\normalsize}
\setbeamerfont{itemize/enumerate subsubbody}{size=\normalsize}
\usecolortheme{seahorse}
\setbeamersize{text margin left=2mm,text margin right=2mm}

\graphicspath{{../../figures/}}

\definecolor{mypine}{rgb}{0.05,0.45,0.05}
\definecolor{mycyan}{rgb}{0.0,0.9,0.9}
\newcommand{\Red}{\textcolor{red}}
\newcommand{\Blue}{\textcolor{blue}}
\newcommand{\Green}{\textcolor{mypine}}
\newcommand{\PineGreen}{\textcolor{mypine}}
\newcommand{\Magenta}{\textcolor{magenta}}
\newcommand{\Cyan}{\textcolor{mycyan}}

\newcommand{\N}{\mathcal{N}}
\newcommand{\R}{\mathbb{R}}
\newcommand{\T}{{\scriptsize^{\top}}}
\newcommand{\D}{\mathcal{D}}
\newcommand{\F}{\mathcal{F}}
\newcommand{\E}{\mathbb{E}}
\newcommand{\V}{\mathbb{V}}
\newcommand{\M}{\mathcal{M}}
\newcommand{\KL}{\mathcal{KL}}
\newcommand{\cut}[1]{}
\newcommand{\trace}{\operatorname{trace}}

\newcommand{\bmu}{{\boldsymbol{\mu}}}
\newcommand{\btheta}{\boldsymbol{\theta}}
\newcommand{\bepsilon}{\boldsymbol{\epsilon}}
\newcommand{\balpha}{\boldsymbol{\alpha}}
\newcommand{\bbeta}{\boldsymbol{\beta}}
\newcommand{\bphi}{\boldsymbol{\phi}}
\newcommand{\bPhi}{\boldsymbol{\Phi}}
\newcommand{\bSigma}{\boldsymbol{\Sigma}}
\newcommand{\bpi}{\boldsymbol{\pi}}
\newcommand{\blambda}{\boldsymbol{\lambda}}

\newcommand{\argmax}{\operatorname{argmax}}
\newcommand{\argmin}{\operatorname{argmin}}
\newcommand{\ci}{{\bot\negthickspace\negthickspace\bot}} % conditional indep.
\newcommand{\neigh}{\operatorname{ne}}
\newcommand{\vectr}[2]{  \left[ \!\!\begin{array}{c} #1 \\
      #2 \end{array} \!\!\right]}
\newcommand{\deff}{\stackrel{\mathrm{def}}{=}}
\newcommand{\deldel}[2]{\frac{\partial #1}{\partial #2}}

\newcommand{\maketilde}{\raisebox{0.4ex}{\tiny $\sim$}}
\newcommand{\bfa}{\bm a}
\newcommand{\bfb}{\bm b}
\newcommand{\bfe}{\bm e}
\newcommand{\bff}{\bm f}
\newcommand{\bfk}{\bm k}
\newcommand{\bfm}{\bm m}
\newcommand{\bfn}{\bm n}
\newcommand{\bfp}{\bm p}
\newcommand{\bfs}{\bm s}
\newcommand{\bfu}{\bm u}
\newcommand{\bfx}{\bm x}
\newcommand{\bfy}{\bm y}
\newcommand{\bft}{\bm t}
\newcommand{\bfv}{\bm v}
\newcommand{\bfw}{\bm w}
\newcommand{\bfz}{\bm z}
\newcommand{\bfA}{\bm A}
\newcommand{\bfI}{\bm I}
\newcommand{\bfK}{\bm K}


\title{Gibbs sampling in TrueSkill}
\author{Carl Edward Rasmussen}
\date{October 28th, 2016}

\begin{document}


\begin{frame}
\titlepage
\end{frame}


\begin{frame}
\frametitle{Key concepts}

\begin{itemize}
\item In TrueSkill the joint distribution is intractable
\item however, both the 
\begin{itemize}
\item performance differences given skills, and
\item skills given performances
\end{itemize}
are tractable.
\item we derive the Gibbs sampling updates in detail.
\end{itemize}
\end{frame}


\begin{frame}
\frametitle{Gibbs sampling for the TrueSkill model}

We have $g=1,\ldots,G$ games where $I_g$: id of Player 1 and $J_g$:
id of Player 2.\\ The outcome of game $g$ is $y_g=+1$ if $I_g$ wins, $y_g=-1$ if $J_g$ wins.\\[1ex]

Gibbs sampling alternates between sampling skills
$\bfw=[w_1,\ldots,w_M]^\top$ conditional on fixed performance
differences $\bft=[t_1,\ldots,t_N]^\top$, and sampling $\bft$
conditional on fixed $\bfw$.
\begin{enumerate}
\item Initialise $\bfw$, e.g.\ from the prior $p(\bfw)$.
\item Sample the \Blue{\emph{performance differences}} from their
  conditional posteriors
\[
\Blue{p(t_g|w_{I_g}, w_{J_g},y_g)}\;\propto\;
\delta(y_g-\operatorname{sign}(t_g))\N(t_g; w_{I_g}-w_{J_g}, 1)
\]
\item Jointly sample the \Red{\emph{skills}} from the conditional posterior 
\[
\Red{p(\bfw|\bft,\bfy)}\;=\;\underbrace{\Red{p(\bfw|\bft)}}_{\N(\bfw; \mu, \Sigma)}
\propto 
\underbrace{p(\bfw)}_{\N(\bfw; \mu_0, \Sigma_0)}
\prod_{g=1}^G 
\underbrace{p(t_g|w_{I_g}, w_{J_g})}_{\propto\;\N(\bfw; \mu_g, \Sigma_g)}
\]
\item Go back to step 2.
\end{enumerate}

\end{frame}


\begin{frame}
\frametitle{Gaussian identities}

The distribution for the performance is both Gaussian in $t_g$ and
proportional to a Gaussian in $\bfw$
\[
\begin{split}
p(t_g|w_{I_g}, w_{J_g})\;&\propto\;\exp\big(-\tfrac{1}{2}(w_{I_g}- w_{J_g}-t_g)^2\big)\\
&\propto\;\N\Big(-\tfrac{1}{2}
\big(\begin{array}{c}
w_{I_g}-\mu_1\\
w_{J_g}-\mu_2
\end{array}\big)^\top
\Big[\begin{array}{r r}
1 & -1\\
-1 & 1
\end{array}\Big]
\big(\begin{array}{c}
w_{I_g}-\mu_1\\
w_{J_g}-\mu_2
\end{array}\big)
\Big)
\end{split}
\]
with $\mu_1 - \mu_2 = t_g$. Notice that
\[
\Big[\begin{array}{r r}
1 & -1\\
-1 & 1
\end{array}\Big]
\big(\begin{array}{c}
\mu_1\\
\mu_2
\end{array}\big)
=
\big(\begin{array}{r}
t_g\\
-t_g
\end{array}\big)
\]

Remember that for products of Gaussians precisions add up, and means
 weighted by precisions (natural parameters) also add up:
\[
\N(\bfw; \mu_a, \Sigma_a)\N(\bfw; \mu_b, \Sigma_b) = z_c \;
\N\big(\bfw; \mu_c,
\Sigma_c \big)
\]
where $\Sigma_c^{-1} = \Sigma_a^{-1}+\Sigma_b^{-1}$ and $\mu_c = \Sigma_c (\Sigma_a^{-1}\mu_a+\Sigma_b^{-1}\mu_b)$.
\end{frame}


\begin{frame}
\frametitle{Conditional posterior over skills given performances}

We can now compute the covariance and the mean of the conditional posterior.
\[
\Blue{\Sigma^{-1}} = \Sigma_0^{-1} + 
\Green{\underbrace{\sum_{g=1}^G \Sigma_g^{-1}}_{\tilde\Sigma^{-1}}} 
\;\;\;\;\;\;\;\;\;\;\;
\mu = \Blue{\Sigma}\big(\Sigma_0^{-1}\mu_0 + 
\Red{\underbrace{\sum_{g=1}^G\Sigma_g^{-1}\mu_g}_{\tilde\mu}}\big),
\]
where each game precision $\Green{\Sigma_g^{-1}}$ contain only 4
non-zero entries. The combined precision is:
\[
\begin{split}
\Green{[\tilde\Sigma^{-1}]_{ii}} 
&= \sum_{g=1}^G \delta(i-I_g)+\delta(i-J_g)\\
\Green{[\tilde\Sigma^{-1}]_{i\neq j}} 
&= - \sum_{g=1}^G \delta(i-I_g)\delta(j-J_g) + \delta(i-J_g)\delta(j-I_g),  
\end{split}
\]
and for the mean we have
\[
\Red{\tilde\mu_i}\;=\;\sum_{g=1}^G t_g\big(\delta(i-I_g)-\delta(i-J_g)\big).
\]
\end{frame}

\begin{frame}
\frametitle{Implementing Gibbs sampling for the TrueSkill model}

we have derived the conditional distribution for the \Blue{performance
  differences} in game $g$ and for the \Red{skills}. These are:
\begin{itemize}
\item the posterior conditional \Blue{performance difference} for $t_g$ is a
  univariate truncated Gaussian. How can we sample from it?
\begin{itemize}
\item by rejection sampling from a Gaussian, or
\item by the inverse transformation method (passing a uniform on an
  interval through the inverse cumulative distribution function).
\end{itemize}
\item the conditional \Red{skills} can be sampled jointly form the
  corresponding Gaussian (using the cholesky factorization of the
  covariance matrix).
\end{itemize}

Once samples have been drawn from the posterior, these can be used to
make predictions for game outcomes, using the generative model.\\
\hfill\Blue{\emph{How would you do this?}}
\end{frame}


\end{document}
